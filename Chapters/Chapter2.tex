\chapter{Approach}
\section{Proposed Model}
A modified scheduling algorithm is proposed that optimizes for employee satisfaction instead of minimizing cost. In the setting of small and medium-sized businesses, improvements in morale may more tangible than the labor costs saved.

Current workforce scheduling algorithms gain these cost savings by scheduling employees down to 15-minute discrete resolution, including breaks. Per the motivating example, the workforces targeted by this algorithm do not require such precise scheduling. Instead, they seek a more basic interface that schedules employees in 4-hour blocks. 

In addition to the minimum employees working per hour that current models  provide, the proposed new model sets a maximum constraint of number of workers per shift. After specifying availability, employees select priority shifts every week. Each shift is assigned a weight on a per-employee basis, according to employee prioritization.

\section{Project Goals}

The goal of this project is to design a scheduling algorithm that takes a 7-day work week and analyzes it as 42 discrete shifts each 4 hours in length. 

The manager sets how many employees are required at each shift throughout the week. If the business is closed, then 0 employees are required. Note, however, that the model should function for a business that never closes. 

The manager also specifies the minimum and maximum shifts each employee must work each week. For instance, if a manager wish to keep someone from exceeding 30 hours of work per week, they specify a maximum of 6 shifts per week (28 hours).

Each employee may set their availability by specifying a boolean 'available to work' or 'unavailable to work' for every shift throughout the week. Then, they are provided with the option to set boolean 'preferred shift' or 'no preference' for every shift where they are available to work. 

To comply with basic labor practices, the model also implements a limit on the number of shifts an employee may work per day. This is may be specifed by the manager, but in general may be treated as 2 shifts (8 hours) per day. 

A feasible solution of the model treats employee availability, work limits, number of employees working per shift, and shifts per 24-hour segment as hard constraints. The objective function seeks to maximize the placement of employees in shifts based on preference.

\section{Model Design}

A binary integer programming (BIP) model was selected because of the discrete and boolean nature of the decision variable, which applies well to shift scheduling. Other scheduling algorithms have been implemented as mixed integer programs, but those models have focuses on complex shift overlaps between employees. Because employees in this model work in integral numbers of shifts with no overlap, such an model does not suitably apply.

Due to the nondeterministic time complexity of a BIP algorithm, heuristics serve an important role in providing quick runtimes and reaching feasibility. By making the decision variable have the same indices as the availability or preference inputs, straightforward heuristics for the first iteration based on these inputs are possible. 

While the BIP model is nonlinear, removing the binary constraint on the decision variable and treating it into a continuous 0-1 interval constraint makes the problem linear. Thus, a linear programming relaxation model may be used to generate the first BIP iteration. Specically, using the Simplex algorithm to solve the linear program, then rounding the continouous variables to discrete binary variables for the first interval provides a solution that is a reasonable first approximation for a feasible solution of the BIP model. The marginal effect on overall runtime by adding the linear programming relaxation is trivial due to the polynomial runtime of the Simplex algorithm.  

A variety of algorithms for solving BIP models have been commercially implemented in a variety of software packages. In general, the NP-Complete classification of BIP problems means that runtimes have nondeterministic polynomial runtime and are computationally difficult.

\section{Implementation}

This model is best implemented in a Software as a Service web platform, where employers are able to manage parameters and employees are able to provide the necessary input remotely. 

First, employers configure the workweek and anticipated load in the system. The minimum and maximum employees per shift are set, with a maximum of 0 meaning that the business is closed. 

Separate models are needed for each role. For instance, separate models are needed for cooks and dishwashers because they are non-overlapping sets.

Finally, employers configure employees and working standards. Specifically, constraints on per-employee hours are set, and minimum and maximum shift lengths are specified. 

\section{Usage}

Constraints in the model are designed so that the schedule is to be repeated every week to comply with labor parameters. When considering that the number of shifts an employee may work in a 24-hour period is limited, the first shift and the last shift of the week are considered to share a boundary, such that if the schedule were repeated every week, work limit parameters would satisfied across this boundary. 

If employee availability and preferences changes on a week-by-week basis, the employee availability matrix may be used to manually enforce worktime parameters without additional modifications to the model. For instance, if the prior workweek ended with three consecutive shifts for a particular employee, they could be marked as 'unavailable' for the first 3 shifts of the proceeding week such that they do not exceed the number of allowed shifts in a 24-hour period.  






